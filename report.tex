\documentclass[a4paper,12pt]{article}

% 导入中文宏包 (必须使用 XeLaTeX 编译)
\usepackage[UTF8]{ctex}

% 页面布局
\usepackage{geometry}
\geometry{left=2.5cm,right=2.5cm,top=2.5cm,bottom=2.5cm}

% 常用宏包
\usepackage{graphicx}   % 图片支持
\usepackage{float}      % 图片浮动位置
\usepackage{hyperref}   % 超链接
\usepackage{amsmath}    % 数学公式
\usepackage{xcolor}     % 颜色支持

% 代码高亮设置
\usepackage{listings}
\definecolor{codegreen}{rgb}{0,0.6,0}
\definecolor{codegray}{rgb}{0.5,0.5,0.5}
\definecolor{codepurple}{rgb}{0.58,0,0.82}
\definecolor{backcolour}{rgb}{0.95,0.95,0.92}

\lstdefinestyle{mystyle}{
    backgroundcolor=\color{backcolour},   
    commentstyle=\color{codegreen},
    keywordstyle=\color{magenta},
    numberstyle=\tiny\color{codegray},
    stringstyle=\color{codepurple},
    basicstyle=\ttfamily\footnotesize,
    breakatwhitespace=false,         
    breaklines=true,                 
    captionpos=b,                    
    keepspaces=true,                 
    numbers=left,                    
    numbersep=5pt,                  
    showspaces=false,                
    showstringspaces=false,
    showtabs=false,                  
    tabsize=2,
    % 关键修复:允许特殊字符,防止中文乱码报错
    extendedchars=false 
}
\lstset{style=mystyle}

% 文档信息
\title{\textbf{2026赛季视觉组选拔考核综合技术报告}\\ \large 任务二:模型训练 \& 附加任务:曝光优化}
\author{考核人:[你的名字]}
\date{\today}

\begin{document}

\maketitle
\tableofcontents
\newpage

% ==========================================
% 第一部分:任务二 YOLO 模型训练
% ==========================================
\section{考核任务二:YOLO 模型训练实战与思考}

\subsection{项目概述}
本项目基于 YOLOv11 框架,完成了针对赛场关键元素(台球 ball、魔方 cube)的目标检测模型训练。项目涵盖了从视频数据清洗、数据集构建、模型训练到推理验证的全流程。

\subsection{实战流程}

\subsubsection{1. 数据集准备 (Data Preparation)}
使用 Python 脚本从原始视频(\texttt{raw.mp4}, \texttt{raw1.mp4}, \texttt{raw2.mp4})中提取训练图像。为了保证数据集的多样性并防止文件覆盖,编写了 \texttt{cut\_final.py} 脚本,支持自定义采样步长和文件名前缀。

\begin{lstlisting}[language=Python, caption=视频抽帧核心代码 (cut\_final.py)]
# Core Logic: Save frame by step and rename
if current_frame % step == 0:
    filename = f"{output_folder}/{image_prefix}_{saved_count}.jpg"
    cv2.imwrite(filename, frame)
    saved_count += 1
\end{lstlisting}

\subsubsection{2. 数据标注 (Labeling)}
使用 \textbf{X-AnyLabeling} 工具进行标注,采用 YOLO 格式导出。
\begin{itemize}
    \item \textbf{类别定义}:
    \begin{itemize}
        \item Class 0: \texttt{ball} (台球)
        \item Class 1: \texttt{cube} (魔方)
    \end{itemize}
    \item \textbf{标注策略}:手动剔除模糊图像,确保每个目标均有独立包围框(Bounding Box)。
\end{itemize}

\subsubsection{3. 数据集划分 (Dataset Splitting)}
编写 \texttt{split\_data.py} 脚本,自动将图像和对应的标签文件按 8:2 的比例随机划分为训练集(train)和验证集(val),并整理为 YOLO 标准目录结构。

\subsubsection{4. 模型训练 (Model Training)}
基于 \texttt{yolo11n.pt} 预训练模型进行迁移学习。训练参数配置如下:
\begin{itemize}
    \item \textbf{Epochs}: 50
    \item \textbf{Batch Size}: 8
    \item \textbf{Image Size}: 640
    \item \textbf{Device}: GPU (CUDA)
\end{itemize}

\begin{lstlisting}[language=Python, caption=训练启动脚本 (train.py)]
from ultralytics import YOLO

if __name__ == '__main__':
    model = YOLO('yolo11n.pt') 
    model.train(
        data='data.yaml', 
        epochs=50, 
        imgsz=640, 
        batch=8,
        workers=0,  # Must be 0 on Windows
        device='0',
        name='task2_result'
    )
\end{lstlisting}

\subsection{思考题:魔方角点识别}

\subsubsection{Q1: 为什么我们需要角点?(Why?)}
目前的 YOLO 模型仅能输出 2D 边界框(Bounding Box),这在机械臂抓取任务中存在局限性:
\begin{enumerate}
    \item \textbf{PnP 解算需求}:PnP (Perspective-n-Point) 算法需要利用物体在世界坐标系中的 3D 点与图像中的 2D 像素点进行匹配,从而求解相机的位姿(旋转矩阵 $R$ 和平移向量 $t$)。角点是进行点对点匹配的最佳特征。
    \item \textbf{姿态估计}:仅凭矩形框无法获知魔方的旋转角度(如平放、斜放)。通过识别角点,可以精确计算魔方的 6DoF 姿态,辅助机械臂规划抓取路径。
\end{enumerate}

\subsubsection{Q2: 如何准确识别角点?(How?)}
针对魔方角点识别,提出以下两种技术路线:

\textbf{方案 A:基于深度学习的关键点检测 (Keypoint Detection)}
\begin{itemize}
    \item \textbf{思路}:利用 YOLO-Pose 等模型,将魔方的 8 个角点定义为“骨骼关键点”进行训练。
    \item \textbf{优点}:端到端输出,鲁棒性强,受光照和背景干扰较小。
    \item \textbf{缺点}:标注成本极高,需要对每个角点进行精确打点。
\end{itemize}

\textbf{方案 B:传统视觉 + 几何约束 (Traditional CV)}
\begin{itemize}
    \item \textbf{思路}:YOLO 裁剪 ROI区域 $\rightarrow$ Canny 边缘检测 $\rightarrow$ 霍夫变换 (Hough Lines) 拟合直线 $\rightarrow$ 计算直线交点。
    \item \textbf{优点}:不需要大量训练数据,推理速度快。
    \item \textbf{缺点}:对边缘磨损和复杂光照敏感,容易产生误检。
\end{itemize}

\newpage

% ==========================================
% 第二部分:附加任务 复杂光照优化
% ==========================================
\section{附加考核:复杂光照下的曝光优化调研}

\subsection{问题背景}
在实际比赛中(尤其是高纬度地区赛场),常出现极端的\textbf{高反差(High Contrast)}光照条件。由于地面反光强烈,相机自动曝光(AE)算法受全图平均亮度误导,降低了曝光时间,导致位于阴影中的暗色目标(黑桶)细节完全丢失。

\subsection{现有算法分析}
传统的曝光检测算法通常基于\textbf{全局亮度均值}或\textbf{全局直方图统计}:
\begin{itemize}
    \item \textbf{缺陷}:当画面中存在大面积高亮区域(如反光地面)时,全局均值被拉高,掩盖了局部暗部过暗的事实。
    \item \textbf{验证}:使用编写的 \texttt{app.py} 工具分析去年的比赛视频截图,发现其亮度直方图极不均匀,且暗部区域对比度极低。
\end{itemize}

\subsection{解决方案调研}

\subsubsection{方案 A:ROI 自动曝光 (硬件层优化) - 推荐}
\textbf{原理}:通过相机 SDK 接口(如 \texttt{SetAutoExposureROI}),指定相机仅根据画面中央区域(通常是目标出现的位置)进行测光,忽略周围高亮地面的影响。
\textbf{优势}:从物理层面增加感光元件的积分时间,获取更多暗部光子信息,信噪比最高。

\subsubsection{方案 B:Gamma 校正 (图像预处理)}
\textbf{原理}:利用非线性变换提升暗部细节,同时抑制亮部过曝。公式如下:
\begin{equation}
    V_{out} = V_{in}^{\gamma} \quad (\text{其中 } \gamma < 1)
\end{equation}
当 $\gamma$ 取 0.5 左右时,可以显著提亮低灰度区域。

\begin{lstlisting}[language=Python, caption=Gamma 校正实现代码]
def apply_gamma_correction(image, gamma=0.5):
    invGamma = 1.0 / gamma
    table = np.array([((i / 255.0) ** invGamma) * 255 
                      for i in np.arange(0, 256)]).astype("uint8")
    return cv2.LUT(image, table)
\end{lstlisting}

\subsubsection{方案 C:自适应直方图均衡化 (CLAHE)}
\textbf{原理}:不同于全局直方图均衡化(HE),CLAHE 将图像划分为多个小块(Tiles)分别进行均衡化,并限制对比度阈值以防止噪声放大。
\textbf{优势}:能有效增强局部纹理细节,非常适合处理“黑桶内部看不清”的问题。

\begin{lstlisting}[language=Python, caption=CLAHE 实现代码]
def apply_clahe(image):
    lab = cv2.cvtColor(image, cv2.COLOR_BGR2Lab)
    L, A, B = cv2.split(lab)
    # ClipLimit limits contrast amplification
    clahe = cv2.createCLAHE(clipLimit=2.0, tileGridSize=(8, 8))
    L = clahe.apply(L)
    return cv2.cvtColor(cv2.merge([L, A, B]), cv2.COLOR_Lab2BGR)
\end{lstlisting}

\subsection{演示系统实现}
为了验证上述算法,开发了基于 Flask 的 Web 演示工具 \texttt{app.py}。
\begin{itemize}
    \item \textbf{诊断功能}:集成亮度直方图、局部对比度分析、连通区域检测等 5 种算法判断图像是否过曝。
    \item \textbf{优化功能}:实时对比原图、线性调整、HE 以及 CLAHE 的处理效果。
    \item \textbf{结论}:实验表明,\textbf{CLAHE} 算法在恢复暗部纹理方面效果最佳,建议与 ROI 自动曝光配合使用。
\end{itemize}

\end{document}